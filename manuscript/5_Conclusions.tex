\section{Conclusions and outlook}\label{conclusions}
%Rapid and equitable decarbonization of the heat sector in buildings is an indispensable cornerstone in a sustainable society. Special attention is needed for the rented residential buildings sector since an investment decision in sustainable technologies is in the property owner's hands. Simultaneously, an expected increase in the CO\textsubscript{2} price primarily impacts the tenant's energy costs. This work studies cost-optimal subsidy payment strategies incentivizing sustainable heat system implementation and retrofitting measures at the multi-apartment building level. We analyze the results of the application of the developed modeling framework to a partly renovated old building switching either to the district heating network or implementing an air-sourced heat pump system under several decarbonization storylines. \added{Thus, the heating system change is implemented against the background of decarbonization of the feeding energy mix for both technology alternatives.}\vspace{0.5cm}
%
%We find that a fair and equitable switch to a sustainable heat system is possible but with massive public subsidy payments. In particular, the property's owner investment grant and additional rent-related revenues derived from the building renovation measures are crucial to trigger the profitability of the investment. At the same time, subsidy payments to the tenants are required at the beginning of the investment period to limit the energy- and rent-related spendings. Furthermore, the results impressively show that the heat pump alternative is not competitive in supplying heat service needs in partly renovated old buildings. Either the subsidy payments are significantly higher than in the district heating case, or the equitability constraints of the model cannot be satisfied. Deep building renovation and associated reduction of heat demand enable feasible solutions but with high total costs. In this case, passive retrofitting measures need to be incentivized, too.\vspace{0.5cm}
%
%Furthermore, the results demonstrate that allocating the costs of inaction between the governance, the property owner, and the tenants is an important lever and can reduce the required subsidy payments. First and foremost, the biggest drop of the total subsidies (to nearly half) takes place when the costs of inaction are completely borne by the property owner. Also, a decrease in the property owner's interest rate reduces the total costs but limits the maximum share of the costs of inaction allocated to the property owner and implies a lower bound of the cost-minimized solution.\vspace{0.5cm}
%
%Future work may investigate a stronger coupling of active and passive building renovation measures as a necessary precondition for subsidy payments. This could bring further insights to decarbonization \added{and public financial} strategies with an eye on the heat demand and sustainable heat source alternatives in the multi-apartment residential building sector (i.e., climate neutrality in 2050). \added{In this context, further in-depth analyses regarding the public financial deficit (i.e., the interaction between governance's subsidy payments and CO\textsubscript{2} tax revenues) should be conducted for different sustainable technology alternatives and retrofitting levels.} Besides, the tenant's diversification within the building could be improved (e.g., different willingness to pay to contribute to CO\textsubscript{2} mitigation). More generally, this study could be extended by introducing further technology options, such as solar PV and heat and electricity storage systems. 
%
%
%
%
