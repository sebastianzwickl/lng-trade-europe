\section{Introduction}
The world is committed to achieving carbon neutrality by mid-century. Undisputed thereby are measures that increase the share of renewable energy in the energy system and thus replace fossil energy sources. However, the speed on the way there and the specific target year in which net zero emissions are emitted vary between regions. China, for example, has defined 2060 as the target year, while Europe aims to achieve climate neutrality in 2050. For these regions and all others, the question arises of how this sustainable energy transition is shaped in concrete terms. The consensus is that transitional solutions and so-called bridge technologies (or bridge fuels) are necessary if renewable energy cannot fully supply the energy system. A pillar of these bridge technologies, namely liquified natural gas (LNG), is the subject of this paper.                

Díe Wichtigkeit von LNG hat sich bisher sehr stark zwischen den Regionen unterschieden. 



\begin{itemize}
	\item bezogen auf europa piped-gas bis vor kurzem
	\item weshalb Europa als LNG Markt bisher uninteressant war 
	\item durch den Angriffskrieg fundamental geändert 
	\item LNG steht nun auf der Agenda in EU
	\item essenziell für die versorgungssicherheit in europa (neue Terminals an den Kusten gebaut)
	\item krisenmodus lng vergleich mit piped gas nicht möglich da es um versorgungssicherheit geht aber gleichwertig mit dem einsatz von öl its emissions
	\item aktuell supply möglich weil china sehr geringen verbrauch hat, daher derzeitige situation nicht repesentativ 
	\item deswegen auch für exportländer attraktiv 
	\item unklar wie sich mittel- bis langfristig equilibirum einstellt bzw. willingess to pay sein wird und 
	\item dazu trägt auch bei dass viele länder im jahre 2022 versucht haben sehr kurzfristige verträge abzuschließen im krisenmodus 
\end{itemize}

the core objective of this work...