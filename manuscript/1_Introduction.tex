\section{Introduction}
The world is committed to achieving carbon neutrality by mid-century. Undisputed thereby are measures that increase the share of renewable energy in the energy system and thus replace fossil energy sources \cite{yuan2022race}. However, the speed on the way there and the specific target year in which net zero emissions are emitted vary between regions. China, for example, has defined 2060 as the target year \cite{jia2021achieve}, while Europe aims to achieve climate neutrality in 2050 \cite{wolf2021european}. For these regions and all others, the question arises of how this sustainable energy transition is shaped in concrete terms \cite{capros2018outlook}. The consensus is that transitional solutions and so-called bridge technologies (or bridge fuels) are necessary if renewable energy cannot fully supply the energy system \cite{gursan2021systemic}. A pillar of these bridge technologies, namely liquified natural gas (LNG), is the subject of this paper.\vspace{0.35cm}                
           
So far, the role of LNG in energy systems has differed significantly among global regions. Traditionally, the Asian market and particularly the Japanese one had a strong focus on LNG. Other countries, for example, China and South Korea, have shifted to LNG and increased their demand partly significantly in the past decades \cite{aguilera2014role}.    



In the context of Europe, LNG's importance was minor since Europe has been supplied by piped-gas in the last decades. Main supply country for European gas demand was Russia, as the share of russian piped-gas was x\% in 20xx. Not only the geographical nähe but also the general cheap price of Russian piped-gas in the past was the main reason why Europe as LNG market has been unattractive until now. This situation has been changed fundamentally as a result of the invasion of the Ukraine by Russia in February 2022. Als Reaktion belegt mit Sanktionen worauf Russland die versorgung eingestellt hat. 
           





\begin{itemize}
	\item bezogen auf europa piped-gas bis vor kurzem
	\item weshalb Europa als LNG Markt bisher uninteressant war 
	\item durch den Angriffskrieg fundamental geändert 
	\item LNG steht nun auf der Agenda in EU
	\item essenziell für die versorgungssicherheit in europa (neue Terminals an den Kusten gebaut)
	\item krisenmodus lng vergleich mit piped gas nicht möglich da es um versorgungssicherheit geht aber gleichwertig mit dem einsatz von öl its emissions
	\item aktuell supply möglich weil china sehr geringen verbrauch hat, daher derzeitige situation nicht repesentativ 
	\item deswegen auch für exportländer attraktiv 
	\item unklar wie sich mittel- bis langfristig equilibirum einstellt bzw. willingess to pay sein wird und 
	\item dazu trägt auch bei dass viele länder im jahre 2022 versucht haben sehr kurzfristige verträge abzuschließen im krisenmodus 
\end{itemize}

the core objective of this work...