\section{Introduction}
The fulfillment of the Paris Climate Agreement \cite{agreement2015paris} and its climate goals pose enormous challenges for future energy system. Concerning the remaining carbon budget of the mentioned climate goals limiting the global average temperature increase below 1.5 and 2.0°C, there is a substantial risk of running out of time to transition to a sustainable carbon-neutral energy system without adequate and rapid actions at all levels \cite{duan2020modeling}. Said levels include, for instance, the energy supply chain, how energy services are delivered, and consumer behavior. Supported by the main findings from scientific work in the last decades, long-term energy system planning is essential, as it is likely the most promising way to achieve carbon-neutral and cost-effective energy systems in the future \cite{auer2020development}. Moreover, against the backdrop of the significant energy crisis in 2022 triggered by the invasion of Ukraine, long-term decisions about energy systems and their impact on ensuring political stability have become even more critical \cite{papadis2020challenges}. Best practices for elucidating the cornerstones of sustainable energy systems are mainly built on model-based integrated analyses. Such (modeling) concepts enable the quantitative envisioning of feasible pathways achieving pre-defined objectives for developing energy systems under various influencing factors and conditions \cite{felder2021review}. These models are, in particular, large-scale energy system models that focus on the long-term decarbonization of energy systems. Their wide-scope vision of energy systems usually determines the optimal investment decision and dispatch of energy generation technologies and infrastructure. However, the quickly increasing complexity and size of the models (e.g., by integrating different sectoral demands such as heat, transport, and others) make the introduction of simplifications concerning different domains necessary, mainly to keep the computation time within reasonable limits \cite{kotzur2021modeler}.\vspace{0.5cm}

The scope of this paper deals in detail with the frequent simplification in the representation of the heat sector introduced in large-scale energy system models. We elaborate here on district heating and its role in large-scale energy system models as this centralized heat supply option is often neglected and not explicitly considered in those models. That is also why most of the model-based decarbonization pathways have difficulties in making robust estimates about the heat supply by district heating. Even if large-scale energy system models consider district heating, it usually results in quantitative values at the country level\footnote{The spatial granularity of large-scale energy system models is often oriented on the different electricity price zones, and thus, a node per country is common.}, which has proven to be an insufficient spatial resolution for analyzing district heating as, at this resolution, for example, a realistic account of infrastructure-related investments is significantly limited or actually even impossible. Undisputedly, most decarbonization pathways show a strong trend toward electrification of the heat sector. However, how much heat demand can (or should) be electrified on the one hand depends on the availability, and the number of sustainable alternatives in both the electricity and heat sector and, on the other hand, has substantial implications on the need for district heating. Therefore, we explore at the heat sector in Norway, particularly that part which is not directly electrified, from the holistic viewpoint of achieving a cost-optimal European decarbonization of the electricity and heat sectors by 2060. We question whether this fact will (have to) lead to a more significant role for district heating in Norway, a country with a significantly high share of electric heat systems so far.\vspace{0.5cm}

Against this background, the core objective is to examine district heating in Norway at the regional levels until 2060. For this, we consider the cost-optimal network expansion and energy technology dispatch of district heating in each of the five Norwegian electricity price zones. We build upon two existing optimization models and combine them to get a single framework given the typical approach to minimize total system costs over time (from the network operator's perspective). We consider the existing district heating infrastructure (i.e., energy generation capacities and network pipelines) as a starting point. Additionally, we introduce tailor-made restrictions and constraints to the model framework about vital determining parameters derived from the cost-optimal solution at the European level of the large-scale energy system model EMPIRE \cite{backe2022empire}. Mainly, these parameters include electricity and carbon emission prices, maximum energy technology capacities, and total heat delivered. Therefore, this work uses a novel approach for testing the implement-ability of cost-optimal but aggregated heat supply from the European as well as country to the district heating level.\vspace{0.5cm}

The paper is organized as follows. Section \ref{stateoftheart} summarizes the current state-of-the-art in literature and outlines the own contribution of this work beyond existing research. Section \ref{methodology} presents the materials and methods developed in this work, including a detailed description of the modeling framework and other empirical settings. Section \ref{results} presents the results of this work. Section \ref{conclusions} discusses the results, concludes the work, and outlines possible future research.