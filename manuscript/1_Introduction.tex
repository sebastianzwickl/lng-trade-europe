\section{Introduction}
The world is committed to achieving carbon neutrality by mid-century. Undisputed thereby are measures that increase the share of renewable energy in the energy system and thus replace fossil energy sources \cite{yuan2022race}. However, the speed on the way there and the specific target year in which net zero emissions are emitted vary between regions. China, for example, has defined 2060 as the target year \cite{jia2021achieve}, while Europe aims to achieve climate neutrality in 2050 \cite{wolf2021european}. For these regions and all others, the question arises of how this sustainable energy transition is shaped in concrete terms \cite{capros2018outlook}. The consensus is that transitional solutions and so-called bridge technologies (or bridge fuels) are necessary if renewable energy cannot fully supply the energy system \cite{gursan2021systemic}. A pillar of these bridge technologies, namely liquified natural gas (LNG), is the subject of this paper.\vspace{0.35cm}                
           
So far, the role of LNG in energy systems has differed significantly among global regions. Traditionally, the Asian market, particularly the Japanese one, firmly focused on LNG. Other countries, for example, China and South Korea, have shifted to LNG and increased their demand partly significantly in the past decades \cite{aguilera2014role}. Today, as China has become the largest LNG importer worldwide, more than half of China's overall natural gas imports are LNG \cite{lngimports}. On the contrary, LNG's imports to Europe were minor since Europe has been supplied with piped gas in the last decades. The leading supply country for Europe's gas demand was Russia. Traditionally, about 40\% of Europe's total natural gas imports were Russian piped gas. In some European countries, such as Germany, to name one of them, dependence on Russian piped gas was even more significant. In 2020, more than 65\% of natural gas demands were covered by imports from Russia \cite{statista_gas}. The geographical proximity between Russia and Europe and the generally low price of Russian piped gas in the past was the main reason why Europe as an LNG market has been unattractive until now. However, this situation has changed fundamentally as a result of the invasion of Ukraine by Russia in February 2022. In response to Russian aggression and the resulting war in Ukraine, Europe has imposed sanctions on Russia. These have led to the collapse of Russian piped gas imports to Europe in 2022 and, consequently, a rethinking of natural gas in Europe. On the one hand, measures were taken to reduce energy and, thus, gas consumption. On the other hand, Europe had to look for alternatives to replace the lack of imports from Russia. In addition to (limited) increased piped gas imports from Norway and other reactions, the main consequence is that LNG is on Europe's agenda now.\vspace{0.35cm}          

In the short term, LNG is essential for the supply security of Europe's energy systems. That is why Europe was willing to pay high prices in 2022, facing the risk of not being able to meet all the natural gas demands otherwise. In order to bring the procured quantities of LNG to Europe and the countries, new LNG terminals across Europe were also built. For example, Germany, Poland, but also Italy and Greece have already built or are currently in the process to built LNG terminals \cite{lng_terminals}. In view of the above, it can be expected that LNG will play an important role in Europe's energy supply not only in the crisis mode of 2022, but also in the medium term. Although European countries have attempted to negotiate short-term supply contracts for LNG, the investments made in LNG terminals and related transport infrastructure point to longer-term planning\footnote{For example, the LNG terminal in Poland mentioned above will not start operations until 2025.}. However, many questions are unclear in this context so far. In addition to uncertainties regarding how far LNG can contribute to the achievement of European and global climate targets and what quantities will be demanded regionally, there is also the significant issue of how a market equilibrium for LNG will develop in the medium to long term. Particularly, the current market situation in 2022 is not representative for future market equilibrium projections as China's LNG demand is considerably low due to effects of Covid measures there.\vspace{0.35cm}    

Against this background, the core objective of this work is to investigate the global LNG market equilibrium until 2040. Thereby, exchanged LNG quantities between the most relevant import and export countries to meet expected demands and resulting regional LNG prices are in the foreground of the analysis. We focus on the European market and its most relevant export countries to cover Europe's demand until 2040. The analysis furthermore allows estimating future LNG price developments until 2040. Latter is not only a main novelty of the present work but can also be seen as a relevant contribution to the literature. LNG prices are often needed for modeling energy systems and are, in those predominantly, an exogenous input parameter. The present values for LNG price trends, especially for those in Europe that consider the absence of Russian pipeline gas, may therefore be of great importance for future work of the scientific community analyzing the trajectory of the European energy system toward carbon neutrality.\vspace{0.35cm}    

The method applied 




%1.3 Applied methods
%The linear optimization model is implemented in Python, using the optimization modeling language
%Pyomo. For the data manipulation and analysis the Python package pandas is used. The objective
%function minimizes the overall LNG import costs, while fulfilling the LNG demand of all importers.
%As a result of the optimization, the optimal distribution of LNG flows from a set of exporters to a
%set of importers is determined and then used to calculate the prices at each importer.
%As input parameters, model uses monthly or yearly LNG import volumes, LNG export capacities
%and LNG break-even prices. Additionally, geographical, technical and economic data necessary for
%calculating the LNG transportation costs are used as input too.
%Finally, the model is solved with the open-source optimizer GLPK.