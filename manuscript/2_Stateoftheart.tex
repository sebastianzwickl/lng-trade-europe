\section{Literature survey and own contribution}\label{stateoftheart}
%In the scope of this paper, plenty of literature might be relevant to provide background information. We restrict ourselves therefore to three sorts of literature without claiming completeness in each of them. We start the literature survey with an overview of the heat sector in Norway, encompassing a rough description of the current situation and particularly its future development until 2060. After that, we comprehensively examine on concepts of modeling district heating in large-scale energy system models. Thereby, we also take up literature focusing on general challenges in modeling the energy systems. We conclude this section with selected studies that examine the economic viability and profitability of district heating systems. Finally, our own contribution concludes this section and presents this work's novelties. 
%
%\subsection{Development of the heat sector in Norway until 2060}
%\cite{rosenberg2013future}
%\subsection{Modeling district heating in large-scale energy system models}
%\subsection{Economic viability of district heating}
%\subsection{Novelties and own contribution}
%
%%% inklusive politische entscheidungen die bisher auch schon getroffen sind (roadmaps, etc. )
%%
%%reduce the heating demand \cite{seljom2011modelling}
%%
%%development from existing 2nd and 3rd generation of district heating to 4th generation of district heating 
%%based on that, reduction of losses and increasing efficiency \cite{kauko2017dynamic}
%%
%%Trends for the key competitiveness numbers of LTDH are provided.
%%Linear density and return water temperature play crucial roles for the transition to LTDH.
%%\cite{nord2018challenges}
%%
%%The study shows that due to a combination of cheap hydropower and high investment costs, it is quite difficult for new generation units to be profitable \cite{gebremedhin2012there}
%%
%%Heat pump and bio-heating as a replacement for direct electric heaters would create a flexible energy system \cite{hagos2014towards}
%%
%%Role of sustainable heat sources in transition towards fourth generation district heating – A review: Challenges faced by integration of sustainable heat sources into fourth generation district heating systems are discussed \cite{jodeiri2022role}
%%
%%The role of sector coupling in the green transition: A least-cost energy system development in Northern-central Europe towards 2050: Electricity-to-heat is a cost-effective way to decarbonise most of the heat sector \cite{gea2021role}
%%
%%Economic incentives for flexible district heating in the Nordic countries: Heat storage was found to be a no-regrets solution for optimising operation and lowering costs in all cases \cite{sneum2018economic}
%%
%%Decarbonization of a district heating system with a combination of solar heat and bioenergy: A techno-economic case study in the Northern European context: Decarbonization of a Nordic district heating network with solar heat and bioenergy \cite{maki2021decarbonization}
%%
%%\cite{backe2021heat}
%%European wind power competes with district heating for building heat supply in Norway.
%%Less Norwegian electricity consumption yields more export of hydropower.
%%
%%4th generation of district heating, Demand side management and seasonal thermal storage allow large cost and emission reductions \cite{egging2021seasonal}
%%
%%High power prices and limited grid capacity increase the economic viability of seasonal storage systems.
%%The levelized cost of heat from the seasonal storage is competitive with the district heating and electricity prices in Norway.
%%\cite{kauko2022assessing}
%%
%%Thermal energy storage (TES) is a key technology for enabling increased utilization of industrial waste heat in district heating \cite{knudsen2021thermal}
%%
%%Including prosumers does usually not reduce the peak heating demand of the grid \cite{kauko2018dynamic}
%%
%%\cite{kirkerud2014modeling}
%%flexibility offered by the Norwegian heat sector under different meteorological conditions.
%%ensuring high levels of flexibility in the heat sector
%%
%%\cite{askeland2019balancing}
%%A shift from electric heating to district heating frees up generation capacity.
%%Results show that district heating can decrease the maximum load on dammed hydropower facilities, thus freeing up capacity for potential export.
%%
%%\cite{gebremedhin2012introducing}
%%district heating can play an important role in reshaping the municipal energy system which is dominated by electricity and fossil fuels.
%%A substantial reduction of CO2 emissions can be achieved by replacing fossil fuels and electricity. There is also a possibility to improve local environment by replacing inefficient wood combustion which gives high values of particulate emissions and other harmful compounds.
%%
%%\cite{hawkey2014district}
%%Using five case studies, we ask why heat network development in the UK takes a relatively piecemeal and fragmented form in comparison with the Netherlands and Norway, countries whose heating sectors are comparable with the UK and where DH provision is limited.
%%the more coordinated market economies of the Netherlands and Norway retain greater capacity for collaboration between energy utilities, localities and states, resulting in stronger foundations for district energy.
%%
%%\cite{tromborg2007impacts}
%%district heating based on bioenergy are profitable at the current energy prices, but policy incentives in terms of grants, subsidies or feed-in systems make it possible to overcome inertia in investments decisions and provide substantial increase in the supply of bioenergy. 
%%
%%\cite{arabzadeh2020deep}
%%Zero-emission scenarios by 2050 for Helsinki city are presented
%%The strategies are mainly based on extensive electrification employing renewable electricity, storage, and sector-coupling strategies.
%%
%%\cite{amoruso2018german}
%%Which policies prevail in Germany and Norway to foster the deployment of energy efficient and decarbonized solutions for residential buildings?
%%
%%Bioenergy is more valuable in the transport sector than the heating sector \cite{hagos2015comparing}
%%
%%Sustainable building renovation in residential buildings: barriers and potential motivations in Norwegian culture. Results from a questionnaire survey with 341 citizens in Trondheim city, Norway, confirm economic issues as the main barrier for SBR with respondents suggesting different forms of financial support to resolve these barriers \cite{jowkar2022sustainable}
%%
%%\subsection{Modeling district heating in large-scale energy system models}
%%A review on key challenges in energy system modelling.
%%general, multi-scale modelling frameworks should be established and considered both in the dimensions of time, space, technology and energy carrier
%%\cite{fodstad2022next}
%%
%%In this paper, we look particularly at models relevant to national and international energy policy
%%addressing among others the growing complexity of the energy system but no district heating included \cite{pfenninger2014energy}
%%
%%sehr ähnliches paper mit titel: A review of current challenges and trends in energy systems modeling \cite{lopion2018review} aber district heating wird wieder nicht als begriff erwähnt
%%
%%dass sie grundsätzlich nicht zusammenpassen die beiden begriffe wenn man von large-scale spricht, auch wenn man sie integrieren will: large-scale district heating system optimization ist ein stadtteil von einer größeren stadt \cite{dorfner2014large}
%%
%%A Review of District Heating Systems: Modeling and Optimization und klassifizieren auch nach der größe sehr detailliert und zeigt sich das large-scale maximal eine größere stadt bedeutet \cite{talebi2016review}
%%
%%A review of modelling approaches and tools for the simulation of district-scale energy systems / that address district-level interactions in energy systems \cite{allegrini2015review}
%%
%%
%%Kriechbaum et al. [11] evaluate three open-source modelling frameworks considering electricity, natural gas and district heating networks under five modelling criteria (modelling scope, model formulation, spatial coverage, time horizon and data) and three grid-specific modelling criteria (level of detail, spatial resolution and temporal resolution) \cite{kriechbaum2018grid}
%%
%%Evaluating the aggregation methods on an existing district heating network in Austria.
%%95\% computational time reduction in fully dynamic simulation by aggregation methods.
%%Discussion on the potentials and limitations of aggregation methods on 4th generation district heating networks.
%%\cite{falay2020enabling}
%%
%%arbeit die district heating auf länderebene abbildet \cite{lund2010role} excluding pipelines daher grid nicht repesentiert
%%
%%This paper presents a novel and integrated model framework to analyze coupled heat and power sectors. Said framework is designed to assess large-scale/multi-national systems such as the European continent’s one. The integral model character derives from a model sequence that first models heat demand at individual district heating network level, then determines the heat supply schedules of individual heat generation units in these networks and finally uses an extended version of a pre-existing large-scale market model, the WILMAR Joint Market Model, for electricity market simulations \cite{felten2020integrated}
%%
%%The question is therefore whether district heating can contribute to ensuring the sustainability of future energy systems? Denmark is used as a case as the country has a high share of district heating and produces 20\% of the electricity with wind power. The analyses are carried out using the electricity market model Balmorel (international perspective) \cite{munster2012role}
%%
%%balmorel special focus on district heating / Heat demand can also be defined in Areas, such that one Area can either depict a single district heating network or an aggregated heat supply and demand from multiple networks \cite{wiese2018balmorel}
%%
%%grundsätzliche large-scale models so adaptiert bzw. angepasst dass sie lokale heat markets / district heating modellieren können "oemof" \cite{hilpert2018open} multi-purpose modelling environment for modelling and analyzing different systems at scales ranging from urban to transnational.
%%
%%there is also a general trend towards more flexible models \cite{lopion2018review}
%%
%%District heating is seen as an important concept to decarbonize heating systems and meet climate mitigation goals. However, the decision related to where central heating is most viable is dependent on many different aspects, like heating densities or current heating structures. An urban energy simulation platform based on 3D building objects can improve the accuracy of energy demand calculation on building level, but lacks a system perspective. Energy system models help to find economically optimal solutions for entire energy systems, including the optimal amount of centrally supplied heat, but do not usually provide information on building level. Coupling both methods through a novel heating grid disaggregation algorithm, we propose a framework that does three things simultaneously: optimize energy systems that can comprise all demand sectors as well as sector coupling, assess the role of centralized heating in such optimized energy systems, and determine the layouts of supplying district heating grids with a spatial resolution on the street level \cite{steingrube2021method}
%%
%%\subsection{the profitability of district heating systems}
%%% zu groß?
%%% ohne assessing in the überschrift
%%
%%This is mainly due to the combined utilization of the storage in a seasonal approach to shift the waste heat from summer to the winter period and as a short term buffer. Thus, the profits from actively participating on the electricity market with the existing combined heat and power (CHP) plants are increased. \cite{kofinger2018simulation}
%%
%%The study also suggests that CO2-emissions, fuel consumption and socio-economic costs can be reduced by expanding district heating \cite{moller2010conversion}
%%
%%Individual heat pumps seem to be the best alternative to district heating. In the short term, heat pumps are to be found at the same level as district heating in terms of fuel efficiency, CO2 emissions and cost. The cost is a little higher close to the district heating system but a little lower in houses further away. In the long-term perspective, in a 100 per cent renewable energy system, the fuel efficiency is high and, with regard to cost, the solution is more or less equal to district heating. However, it is highly dependent on the distance to existing district heating grids \cite{lund2010role}
%%
%%
%%Power Systems Flexibility from District Heating Networks / evaluate the impact of the DHN on the power system \cite{mitridati2018power}
%%
%%
%%
%%
%%
%%%\subsection{Progress beyond state-of-the-art}\label{novelties}
%%%Based on the literature review, the scientific contribution and novelties of this paper can be summarized as follows:
%%%\begin{itemize}
%%%	\item An equitable and socially balanced change of a currently gas-based heating system toward a sustainable alternative in a rented multi-apartment old building is modeled considering the complex ownership structure and relations between the property owner and tenants to "take action".
%%%	\item Since the governance's first and foremost aim is \replaced{that}{for} the heat system exchange in the multi-apartment building takes place, it is shown \deleted{in }how the governance incentivizes the sustainable investment through monetary and regulative support for both the property owner and tenants. While respecting the property owner's and tenants' individual financial interests, the governance's optimal financial support strategy puts particular emphasis on the highly efficient provision of the residential heat service needs, heat demand reduction, and building efficiency improvements.
%%%	\item The developed analytical framework determines a cost-optimal and socially balanced governance’s subsidization strategy for the decarbonization of the heat demand at the building level. That includes, among others, the profit-oriented behavior of the property owner and the tenants, as well as the abovementioned financial support parity among both sides.  Especially, the proposed optimization model allows detailed quantitative analyses of justice in low-carbon residential buildings and the heating sector with an eye on the complex ownership structure within buildings. Moreover, this work focuses on the economic trade-offs between different agents in the energy transition, particularly the government’s role in triggering private investments and social balance with an eye on the costs of inaction (opportunity costs) and increasing carbon prices.	
%%%	\item Different sensitivity analyses play a key role in this paper, understanding that the impact of varying allocations of the costs of inaction among the governance, the property owner, and the tenants can be seen as one of the main novelties of this work. Moreover, the importance of building stock renovation in the context of public monetary payments is critically discussed. Insights in that respect can help build a more reliable understanding of a sustainable future urban society predominantely living in highly efficient rental apartments.
%%%\end{itemize}